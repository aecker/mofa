\documentclass[letterpaper,12pt]{article}
\usepackage{graphicx}	
\usepackage{natbib}	
\usepackage{amsmath}	
\usepackage{amsfonts}	
\usepackage{bbm}	%
\usepackage[left=3cm,top=3cm,right=3cm]{geometry}

\renewcommand{\topfraction}{0.85} \renewcommand{\textfraction}{0.1} 
\parindent=0cm

\newcommand{\mofa}{\texttt{mofa}}
\newcommand{\Mofa}{\texttt{Mofa}}
\newcommand{\nyu}{$^1$}
\newcommand{\mpia}{$^2$}
\newcommand{\sdss}{{\it SDSS}}
\newcommand{\vect}[1]{\boldsymbol{#1}}
\newcommand{\data}{\vect{x}}
\newcommand{\latent}{\vect{s}}
\newcommand{\mean}{\vect{\mu}}
\newcommand{\cov}{\vect{\Sigma}}
\newcommand{\eye}{\vect{I}}
\newcommand{\bta}{\vect{\beta}}
\newcommand{\lmda}{\vect{\Lambda}}
\newcommand{\ps}{\vect{\Psi}}

\usepackage{url}

\title{Notes on Mixtures of Factor Analyzers}
\author{Ross Fadely\nyu\\
\\
\small
\nyu Center for Cosmology and Particle Physics,
                        Department of Physics, New York University,\\ \small
                        4 Washington Place, New York, NY, 10003, USA \\
}
\begin{document}
\maketitle

\begin{abstract}
A mixture of factor analyzers is a mixture-of-Gaussians model for a
distribution function, but in which each Gaussian has a restricted
number of degrees of freedom: Each component's variance tensor is the
sum of a general low rank tensor and a diagonal (axis-aligned) tensor.
Here we discuss the model, fitting the model to an ensemble of samples
using the expectation-maximization algorithm.  We also introduce
\mofa, a general, open-source, Python implementation.
\end{abstract}

\section{Generalities}

A well known, unsupervised model for multi-variate, multi-modal distributions is the 
\emph{Mixture of Gaussians} (MOG) approach.  The model postulates that the data can be well 
described by a set of $K$ Gaussians, for which each component $k$ stipulates a 
likelihood for data in the component as $N(\mean_k,\cov_k)$.  The mixture of 
components weights each Gaussian by a prior probability $\pi_k$, which act as 
global amplitudes or weights for the different components.  An individual datum $i$ is 
modeled with probability 
\begin{eqnarray}\displaystyle
p(\data_i) = \sum_k \pi_k p(\data_i|\mean_k,\cov_k)\quad ,
\end{eqnarray}
and the total likelihood of the model is simply the product over the $N$ individual likelihoods.  
As we will see later, a useful quantity to define is the \emph{responsibilities} 
\begin{eqnarray}\displaystyle
\label{eqn:rs}
\gamma_{ik} = p(k|\data_i,\mean_k,\cov_k) &=& \frac{p(\data_i|k,\mean_k,\cov_k)\,p(k)}{p(\data_i)} \\
 &=& \frac{\pi_k p(\data_i|\mean_k,\cov_k)}{\sum_k \pi_k p(\data_i|\mean_k,\cov_k)}
\quad ,
\end{eqnarray}

While a powerful generalized model, with $\data_i \in \mathbb{R}^D$ the number of 
free parameters of the model is $KD(D+3)/2$ and may be costly to compute due to 
the inversion of $D\times D$ size covariances $\cov_k$.  \\

Let us re-write the problem in terms of a mixture of latent linear variables $\latent_k$, such 
that $\data = \lmda_k \latent_k + \mean_k + \mathbf{\epsilon}$, where $\latent \sim N(0,\eye)$ and 
$\mathbf{\epsilon} \sim N(0,\cov_k)$.  In this definition, we now have 
$\cov_k = \lmda_k\lmda_k^T + \ps_k$,  that is the covariance for the $k-$th 
component can be thought of as a diagonal variance (we define as) $\ps_k$ and 
a covariance matrix $\lmda_k\lmda_k^T$.  For the MOG case, the latent variables 
$\latent_i \in \mathbb{R}^D$ are precisely the eigenvectors of $\cov_k$, with $|\lmda_k|$ 
 close to the eigenvalues (but slightly different since we have separated the diagonal $\ps_k$).  \\

For \emph{Mixtures of Factor Analyzers} (MOFA), we consider the case where 
$\latent_i \in \mathbb{R}^M$ with $M<D$.  In other words, we consider latent variables that are of 
lower dimensionality than the data, and should be thought of as a lower dimensional projection 
of the data into a subspace with $M$ dimensions.  Here the matrices $\lmda_k$ are a 
$M\times D$, low rank component of $\cov_k$ while $\ps_k$ is the high-rank (diagonal) variances 
of the component.  The reduced rank of $\lmda_k$ has the practical advantage of being faster 
to compute, making MOFA a suitable model when $K,D,N$ are large.  In real-world situations, the 
lower dimensionality of the MOFA model is justified --- often the first $M$ eigenvectors of $\cov$ 
dominate, with the remaining $D-M$ eigenvectors closely represented by diagonal noise ($\ps$).

\section{Expectation-Maximization for MOFA}

Let us define -- \\ \\
\begin{tabular}{lcl}
$K$ &:&the number of mixture components, subscripted by $k$\\
$M$&:& the number of latent dimensions\\
$D$&:& the dimensionality of the data's features\\
$N$ &:&the number of data points\\
$\data$ &:&data which is a $D\times N$ dimensional matrix\\ 
$\latent_k$ &:&the latent projections of $\data$, dimension $M\times N$\\
$\lmda_k$ &:&the transformation matrix for $\latent_k$, dimension $M\times D$\\
$\ps_k$ &:&the diagonal variance matrix for component $k$, dimension $D\times D$\\
\end{tabular}

\subsection{Expectation}

We want to estimate the latent projections and their covariances, under the current model 
parameters for the transformation $\lmda_k$ and $\ps_k$ -- \\ \\

\begin{eqnarray}\displaystyle
\bta_k & = & \lmda_k^T(\lmda_k\lmda_k^T+\ps_k)^{-1}\\
\langle\latent_k|\data,\lmda_k,\ps_k\rangle & = & \bta_k (\data-\mean_k) \\
\langle\latent_k\latent_k^T|\data,\lmda_k,\ps_k\rangle & = & \eye -  \bta_k\lmda_k + \bta_k(\data-\mean_k)(\data-\mean_k)^T\bta_k^T
\quad ,
\end{eqnarray}

In the notation here, $\latent_k$ is a $M\times N$ matrix so it is worth noting that 
$\langle\latent_k\latent_k^T|\data,\lmda_k,\ps_k] $ has dimension $M \times M \times N$.  
Note, $\bta_k$ can be computed making use of a inversion lemma\footnote{
$(\lmda\lmda^T+\ps)^{-1} = \ps^{-1} - \ps^{-1}\lmda(\eye + \lmda^T\ps^{-1}\lmda)^{-1}
\lmda^T\ps^{-1}$, where subscript $k$ has been dropped}.  In addition to computing the 
above, for each $k$ it is also necessary to compute the responsibilities $\gamma_{ik}$ 
(see Eq. \ref{eqn:rs}).

\subsection{Maximization}

It can be shown \citep[for instance,][]{ghahramani96} that the expected log-likelihood for a mixture 
of factor analyzers is :

\begin{eqnarray}\displaystyle
E(p(\data)) &=& c - \frac{1}{2}\sum_{ik} N\log|\psi_k| + \gamma_{ik}\data_i^T\ps_k^{-1}\lmda_k \langle\latent_k|\data,\lmda_k,\ps_k\rangle \\
&& - \frac{1}{2}\sum_{ik} \gamma_{ik}{\rm trace}(\lmda_k^T\psi^{-1}\lmda_k \langle\latent_k\latent_k^T|\data,\lmda_k,\ps_k\rangle )
\quad ,
\end{eqnarray}

By taking derivatives and setting them equal to zero, we get the update steps --

\begin{eqnarray}\displaystyle
\mean_k &\leftarrow& \frac{\sum_i\gamma_{ij}(\data_i-{\lmda_j \langle\latent_k|\data,\lmda_k,\ps_k\rangle})}{\sum_i\gamma_{ij}} \\
\lambda_k &\leftarrow& \left[ \sum_i\gamma_{ij}(\data_i- \mean_k)\langle\latent_k|\data,\lmda_k,\ps_k\rangle) \right]  \left[ \sum_i\gamma_{ij}\langle\latent_k\latent_k^T|\data,\lmda_k,\ps_k\rangle) \right]^{-1}\\
\label{eqn:psik}
\ps_k &\leftarrow& {\rm diag}(\frac{\sum_i\gamma_{ij}\left[(\data-\mean_k)(\data-\mean_k)^T-\lmda_k \langle\latent_k|\data,\lmda_k,\ps_k\rangle(\data-\mean_k)^T\right]}{\sum_i\gamma_{ij}}) \\
\pi_k &\leftarrow& \frac{1}{N}\sum_i\gamma_{ij}
\quad .
\end{eqnarray}

One important distinction here is that we allow the $\ps_k$ matrices to be different for 
different components.  This differs from what is show in a lot of the standard literature, 
like \citet{ghahramani96} which seek to make $\ps_k$ the same for all $k$.  This update 
typically is written as --

\begin{eqnarray}\displaystyle
\label{eqn:psi}
\ps &\leftarrow& \frac{1}{N}{\rm diag}(\sum_i\gamma_{ij}\left[(\data-\mean_k)(\data-\mean_k)^T-\lmda_k \langle\latent_k|\data,\lmda_k,\ps_k\rangle(\data-\mean_k)^T\right]) 
\quad .
\end{eqnarray}

Thus, in our code, if we want to `lock' the $\ps_k$ values to be the same (map Eq. 
\ref{eqn:psik} to Eq. \ref{eqn:psi}) we can just re-weight be the responsibilities 
over $k$ --

\begin{eqnarray}\displaystyle
\ps &\leftarrow& \frac{\sum_k \gamma_k \ps_k}{\sum_k \gamma_k } 
\quad .
\end{eqnarray}

Finally if we want to do probabilistic PCA, all the elements along $\ps$ are the same, 
in which case we take the average value of the elements.

\section{Python Implementation}

Here are a list of the functions and variables in the \Mofa\ class of
the \mofa\ module, with some description of what's going on.
$\star\star\star$ To be updated along with the
code. $\star\star\star$.

\subsection{Variables}

\texttt{}

\end{document}
